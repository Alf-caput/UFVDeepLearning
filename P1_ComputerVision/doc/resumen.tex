\section*{Resumen}

\noindent
El presente trabajo consiste en desarrollar modelos de \textit{Deep Learning} para la clasificación de fauna marina a partir de imágenes. Para ello,
se utilizarán redes neuronales convolucionales profundas (\textit{Convolutional Neural Networks}, CNNs), empleando un \textit{dataset} de acuarios
que contiene fotografías de diversas especies, tales como peces, medusas, pingüinos, tiburones, frailecillos, mantarrayas y estrellas de mar. 

\quad

\noindent
El objetivo principal es diseñar un \textit{pipeline} que permita identificar y clasificar estas especies con alta precisión. Para lograrlo, se 
implementarán técnicas avanzadas como el \textit{transfer learning}, \textit{fine tuning} y \textit{data augmentation}, optimizando el rendimiento del modelo mediante 
el ajuste de hiperparámetros y el uso de arquitecturas modernas. \cite{chollet2021deep}

\quad

\noindent
Además, se evaluará el desempeño del modelo mediante métricas estándar como la precisión, la sensibilidad y la especificidad.

\subsection*{Palabras Clave}

\noindent
\textit{Deep Learning}, Redes Neuronales Convolucionales, \textit{Transfer Learning}, clasificación
