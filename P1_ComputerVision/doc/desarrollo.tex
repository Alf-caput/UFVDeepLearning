\section{Desarrollo}

\subsection{Descripción del \textit{dataset}}

\noindent
El \textit{dataset} con el que estamos trabajando posee tres carpetas: una para los datos de \textit{train}, otra para los datos de \textit{test} y, por último, una 
tercera para los datos de \textit{validation}. En estas carpetas se encuentran las imágenes y un documento en formato \texttt{.csv} llamado \texttt{annotations}, que 
contiene la información de las imágenes. La siguiente tabla describe los atributos del documento


\quad

\begin{figure}[H]
    \centering
    \begin{tabular}{|c|c|c|}
        \hline
        \textbf{Atributo} & \textbf{Tipo de dato} & \textbf{Descripción del atributo} \\ \hline
        filename & string & Nombre de la imagen \\ \hline
        width & int & Anchura de la imagen \\ \hline
        height & int & Altura de la imagen \\ \hline
        class & string & Clase de la imagen \\ \hline
        xmin & \multirow{4}{*}{int} & \multirow{4}{*}{\shortstack[l]{Coordenadas de la clase\\ \text{ }detectada en la imagen}} \\ 
        \cline{1-1}
        ymin &  &  \\ 
        \cline{1-1}
        xmax &  &  \\ 
        \cline{1-1}
        ymax &  &  \\ 
        \hline
    \end{tabular}
    \caption{Atributos del \textit{dataset}}
\end{figure}

\noindent
En cuanto a la distribución de las imágenes, tenemos 448 imágenes de \textit{train}, 63 imágenes de \textit{test} y 127 imágenes de \textit{validation}. Por tanto, 
estamos ante un \textit{dataset} bastante limitado, teniendo en cuenta que estamos trabajando con imágenes y que nuestro objetivo es clasificarlas. Esto abre la puerta 
al uso de técnicas de \textit{data augmentation}, así como al empleo de arquitecturas de red preentrenadas para aplicar \textit{transfer learning}. 

\quad

\noindent
Además, existe un notable desbalanceo en la cantidad de imágenes representativas de cada clase, lo cual refuerza la necesidad de implementar una estrategia adecuada de 
\textit{data augmentation} para mejorar la diversidad del conjunto de datos y mitigar los efectos negativos del desequilibrio en el entrenamiento del modelo.

\subsection{Organización del \textit{dataset}}

\noindent
Para facilitar el trabajo con el \textit{dataset}, hemos automatizado su descarga, extracción y organización en carpetas según la clase dominante de
cada imagen. Para ello, subimos el archivo \texttt{.zip} a \href{https://drive.google.com/uc?id=1iGBv-VT5mm1RiouD-U2qWcU3BYqp2OwE}{Google Drive} y 
desarrollamos un código para descargarlo y descomprimirlo.  

\quad  

\noindent
Luego, con el archivo \texttt{annotations}, identificamos la clase dominante de cada imagen según su área y frecuencia en la imagen, almacenándolas
en carpetas con su respectivo nombre.  


\subsection{Preprocesamiento}

\noindent


\subsection{\textit{Data Augmentation}}

\subsection{\textit{Pre-Train}}

\subsection{\textit{Fine-Tuning}}

\noindent
Podemos observar 


