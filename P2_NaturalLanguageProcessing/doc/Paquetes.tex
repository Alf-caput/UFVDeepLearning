% Paquetes Latex

\usepackage[spanish]{babel} % Idioma y codificación, para poder usar
\usepackage[utf8]{inputenc} % tildes, eñes, etc., sin problemas
\usepackage{float} % Controlar posiciones de figuras
\usepackage[letterpaper,top=2.5cm,bottom=2.5cm,left=3cm,right=3cm]{geometry}
\usepackage{amsmath} % Manejo de ecuaciones
\usepackage{graphicx} % Manejo de imágenes
\usepackage[colorlinks=true, allcolors=blue]{hyperref} % URL a páginas web
\usepackage{parskip} % Separación entre párrafos
\usepackage{graphicx} % Gráficos
\usepackage{float}

\usepackage{biblatex} % Bibliografía
\addbibresource{Bibliografia.bib} % Archivo fuente de bibliografías
\usepackage{csquotes} % Evita warning biblatex

% Justificar
\usepackage{ragged2e}
% Para tablas
\usepackage{tabularx}
\usepackage{booktabs}
\usepackage{array}
\usepackage{multirow}
\usepackage{longtable}
% \PassOptionsToPackage{hyphens}{url}\usepackage{hyperref}
