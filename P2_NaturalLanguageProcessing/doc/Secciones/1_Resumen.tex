\section{Resumen}

Esta práctica explora cómo el tamaño de los vectores de embeddings influye en la identificación contextual de 
palabras, utilizando redes neuronales profundas. Los embeddings, como representaciones vectoriales densas, permiten 
capturar relaciones semánticas y contextuales entre términos, lo que resulta fundamental para tareas de 
procesamiento del lenguaje natural como la clasificación de texto o el análisis de sentimientos. En particular, las 
redes neuronales desarrolladas en este trabajo se aplican al análisis de los sentimientos expresados en comentarios 
de redes sociales, permitiendo identificar de manera automática la polaridad emocional -positiva, negativa o neutra-
de los mensajes publicados por los usuarios. De este modo, se evalúa cómo la dimensionalidad de los embeddings 
afecta la capacidad de los modelos para comprender e interpretar el contenido emocional de los textos
