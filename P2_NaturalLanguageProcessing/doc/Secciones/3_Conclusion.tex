\section{Conclusión}
En conclusión, el trabajo realizado ha permitido profundizar en la aplicación de técnicas de procesamiento de lenguaje natural 
basadas en redes neuronales profundas para el análisis de sentimientos. A través de una cuidadosa preparación y preprocesamiento 
del corpus, así como la implementación de una función personalizada de skip-grams, se ha logrado construir un pipeline robusto y
eficiente para la generación de embeddings semánticos. Estos embeddings, entrenados específicamente sobre el dominio de los textos 
analizados, han demostrado ser capaces de capturar tanto relaciones contextuales locales como patrones globales en el lenguaje, lo
que resulta fundamental para tareas de clasificación de sentimientos. El enfoque adoptado ha evidenciado la importancia de adaptar
y personalizar las herramientas y técnicas existentes a las particularidades del corpus y los objetivos del proyecto, como las dimensiones
de los embeddings y la arquitectura de la red neuronal.